\documentclass[a4paper,12pt]{article}
\usepackage{amsmath}% http://ctan.org/pkg/amsmath
\usepackage{amssymb}
\usepackage{amsthm}
\usepackage{mdframed}
\usepackage[vietnamese]{babel}
\usepackage{mathtools}
\usepackage{listings}% http://ctan.org/pkg/listings
\usepackage{enumitem}
\usepackage{import}
\usepackage{xifthen}
\usepackage{pdfpages}
\usepackage{transparent}
\usepackage{float}
\usepackage{hyperref}
\newtheorem{theorem}{Theorem}[section]
\newtheorem{corollary}{Corollary}[theorem]
\newtheorem{lemma}[theorem]{Lemma}
\newtheorem*{remark}{Remark}
\newtheorem{definition}{Definition}[section]
\newenvironment{ftheorem}
  {\begin{mdframed}\begin{theorem}}
  {\end{theorem}\end{mdframed}}
\newenvironment{fcorollary}
  {\begin{mdframed}\begin{corollary}}
  {\end{corollary}\end{mdframed}}
\newenvironment{flemma}
  {\begin{mdframed}\begin{lemma}}
  {\end{lemma}\end{mdframed}}
\newenvironment{fremark}
  {\begin{mdframed}\begin{remark}}
  {\end{remark}\end{mdframed}}
\newenvironment{fdefinition}
  {\begin{mdframed}\begin{definition}}
  {\end{definition}\end{mdframed}}

\hypersetup{
    colorlinks=true,
    linkcolor=blue,
    filecolor=magenta,      
    urlcolor=cyan,
    pdftitle={Overleaf Example},
    pdfpagemode=FullScreen,
    }
\urlstyle{same}
\newcommand{\incfig}[1]{%
    \def\svgwidth{\columnwidth}
    \import{./Figures/}{#1.pdf_tex}
}

\title{Một vài giới thiệu về Không Gian Banach}
\date{}
\author{Long Nguyen}

\begin{document}
\maketitle

\section{Giới thiệu}

Tất cả chúng ta đều đã rất quen với một không gian, ký hiệu bằng chữ \(\mathbb{R}\). Đây là
"tập số thực".
Có lẽ chúng ta đã quen tới mức không nghĩ nhiều về cấu trúc của những con số này.
Dùng là dùng thôi.

Nếu như nhìn kỹ vào, thì trường số thực mà chúng ta đang đứng trên thì lại là một đỉnh núi!
Chúng ta đang sử dụng một công cụ rất "hoàn thiện" và tiện lợi. Một cách nôm na thì vật thể
\(\mathbb{R}\) này mà chúng ta đang dùng giống như một chiếc ô tô vậy.
Chúng ta sống ở trong đô thị, một chiếc ô tô hoàn toàn có thể đưa chúng ta tới bất kỳ đâu
ta muốn.

Thế nhưng, đôi khi ta muốn lên trên núi (như đi phượt Hà Giang chẳng hạn), hay thi thoảng
ta muốn đi chơi trên sông, hoặc đôi khi chỉ là đi chơi phố cổ Hà Nội thôi, thì bỗng dưng
chiếc ô tô lại không thể sử dụng được nữa.

Nếu như bạn đã từng đi phươt Hà Giang rồi thì chắc chắn bạn sẽ biết rằng, đi bằng ô tô,
bạn đang bỏ lỡ rất nhiều cảnh đẹp. Đi ô tô trên sông thì chắc chắc là không thể rồi.
Và đi ở trong phố cổ Hà Nội thì không gì tiện hơn là một chiếc xe máy để luồn lách,
và để có thể gửi xe một cách thảnh thơi.

Vậy thì cái cấu trúc của không gian số thực mà chúng ta đang dùng cũng vậy. Trong cuộc sống
đời thường, những gì chúng ta vẫn sử dụng hoàn toàn không có vấn đề gì. Nhưng nếu như
công việc đòi hỏi chúng ta phải "lên núi", "đi sông", ví dụ như anh chị em làm khoa học
dữ liệu, hay AI, ML chẳng hạn, đôi khi chúng ta cần có những công cụ phù hợp với hoàn cảnh
cụ thể mà chúng ta đang rơi vào.

Lúc này các bạn có thể tự đặt ra câu hỏi rằng ứng dụng như thế nào? Tôi chỉ làm việc với
các con số thực và máy tính cũng chỉ hiểu số 0 và số 1 thôi mà? Tôi không hình dung nổi được
trường hợp nào cần phải sử dụng những công cụ mà bạn đang nói.

Vậy thì hãy để cho tôi giới thiệu tiếp cho các bạn.
\pagebreak

\begin{remark}
  Tôi viết bài này bằng \LaTeX, nên là một số nơi các bạn sẽ nhìn thấy các ký hiệu
  tiếng Anh. Đây là do được tự động sinh ra.

  Thứ hai là các định nghĩa, bổ đề, định lý, hệ quả, mệnh đề, ghi chú, ... sẽ được
  viết bằng tiếng Anh, đơn giàn là bởi vì tôi được học các khái niệm này bằng tiếng
  Anh và không biết khá nhiều thuật ngữ trong tiếng Việt.
\end{remark}

\section{Giới thiệu khó hơn}

\subsection*{Một ví dụ về không gian}
Tiếp theo ở phía trên, tôi sẽ đưa ra một ví dụ về không gian, nhưng không phải của số.
Vì chắc là anh chị em làm việc với AI/ML sẽ quan tâm tới những gì tôi viết hơn nên đây
là một ví dụ liên quan tới Machine Learning.

Trong ML, việc huấn luyện mô hình có thể được mô tả bằng bài toán như sau:

\begin{mdframed}
  Với một bộ dữ liệu \(\mathcal{D} = \{ (x_1, y_1), (x_2, y_2), \ldots, (x_n, y_n) \}\)
  (có một phân phối \(P(x, y)\) nào đó, nhưng khía cạnh này tôi sẽ không nói ở đây),
  ta cần tìm một hàm số \(h\) nào đó nằm trong \textbf{không gian các hàm số} \(\mathcal{H}\)
  sao cho hàm số \(h\) này `tốt' nhất có thể.

  `Tốt' ở đây có nghĩa rằng, với các điểm dữ liệu nằm trong tập \(\mathcal{D}\), ta có
  \[
    h(x) \approx y
  \]
  và ta muốn giá trị
  \begin{equation}\label{eq:eq-1}
    \mathbb{E}[L(h(x), y)]_{(\overline{x} , \overline{y} ) \sim P(x , y)}
  \end{equation}
  nhỏ nhất có thể.

  Giải thích dễ hiểu hơn thì là ta mong muốn \emph{giá trị kỳ vọng}
  của \emph{loss} của model là nhỏ nhất có thể. Dễ hiểu hơn nữa là model "đoán" càng gần
  giá trị thực tế càng tốt.
\end{mdframed}

Điều mà tôi muốn tập trung vào chính là
\textbf{không gian các hàm số} \(\mathcal{H}\) ở trên. Khi làm việc với ML, ngoài việc
phân tích về dữ liệu thì ta cũng có thể phân tích về chính bản thân các model. Vậy thì ở trong
không gian các model này, ta cần phải có những công cụ khác với những thứ ta vẫn sử dụng hàng ngày.

Một ví dụ đơn giản: Làm thế nào để ta định nghĩa được hai hàm số là ``gần nhau''?
Hàm số \(f(x) = x^2\) và \(g(x) = x^2 + 1\) có ``gần nhau'' hơn so với \(f(x) = x^2\) và
\(h(x) = x^3\)? Các công cụ có được trong quá trình nghiên cứu về không gian Banach
(đúng hơn là giải tích hàm - Functional Analysis) sẽ cho ta câu trả lời và các công cụ cần thiết

Và như vậy chúng ta quay trở lại

\subsection*{Toán}
Trước tiên thì tôi cần phải cho các bạn nhìn thấy được đằng sau lớp sơn bóng bẩy của chiếc
``ô tô'' \(\mathbb{R}\) thì đằng sau ký hiệu đó là gì.

Về mặt toán học, ta có thể nói \(\mathbb{R}\) là một \textbf{không gian Banach} (Banach space).

\begin{fdefinition}
  A \textbf{Banach space} is a complete normed vector space.
\end{fdefinition}

Oke nghe rất xịn xò, ``Banach space'', ``normed vector space'', ``complete''? Hoàn hảo?

Tôi sẽ giải thích các khái niệm này cho các bạn thông qua

\section{Một số định nghĩa cơ bản}

Từ khoá đầu tiên là ``vector space''. Một không gian vector \(X\) ở trên trường \(F\) là một
tập hợp \(X\) cùng với hai phép toán tử, thoả mãn 8 tiên đề, v.v. Nếu đi sâu hơn thì ta
đang học môn cấu trúc đại số mất.

Hiện tại, chỉ cần hiểu không gian vector (không gian) là một tập hợp (số).

Từ khoá quan trọng đầu tiên mà ta quan tâm là ``norrmed space''.
Vậy thì ``norm'' (chuẩn) là gì?

\begin{fdefinition}
  A \textbf{norm} on a vector space \(X\) over a field \(\mathbb{F}\) is a mapping \(\| \cdot \|: X \to \mathbb{R}\) such that
  \begin{enumerate}
    \item $\| x \| \geq 0$ for all $x \in X$ and $\| x \| = 0$ if and only if $x = 0$.
    \item $\| \alpha x \| = |\alpha| \| x \|$ for all $\alpha \in \mathbb{F}$ and $x \in X$.
    \item $\| x + y \| \leq \| x \| + \| y \|$ for all $x, y \in X$.
  \end{enumerate}
\end{fdefinition}

Trước tiên, \emph{norm} cho chúng ta một ý niệm về "khoảng cách" giữa hai "điểm" bất kỳ
trong một "không gian" bất kỳ. (Tôi không đưa ra khái niệm về "khoảng cách" / "metric" ở đây,
nhưng nó cũng khá là tương tự). Khoảng cách giữa hai điểm \(x\) và \(y\) được suy ra từ norm
chính là
\[
  d(x, y) = \| x - y \|
\]

Nhìn vào định nghĩa này, có thể thấy nó rất\dots hiển nhiên. Một phần là bởi vì định
nghĩa này đã được ``thiết kế'' để cho khái niệm thông thường về khoảng các vẫn được giữ nguyên.

Tuy nhiên thì nó có thể được khái quát lên hơn rất nhiều. Hãy quay trở lại ví dụ ở trên,
một cách phổ biến để đo khoảng cách giữa các hàm số chính là bằng \emph{uniform norm}.

\begin{fdefinition}
  Let \(X\) be a set. The \textbf{uniform norm} on \(X\) is defined by
  \[
    \| f \|_{\infty} = \sup_{x \in X} |f(x)|
  \]
\end{fdefinition}

Từ khoá cuối cùng mà chúng ta cần phải khám phá đó chính là ``complete''.

Vậy thì một không gian phải như thế nào mới là một không gian ``hoàn chỉnh''?

\begin{fdefinition}
  A vector space \(X\) is \textbf{complete} if every Cauchy sequence in \(X\) converges to a limit in \(X\).
\end{fdefinition}

Oke, vậy là định nghĩa này lại sinh ra một loạt các từ khoá khác mà chúng ta lại phải khám phá.
Tôi sẽ nói cho các bạn biết rằng những định nghĩa này tuy rằng trông có vẻ rất ngắn và đơn giản,
nhưng cách sử dụng chúng thì lại biến hoá không lường. Bạn có thể tin tôi vì tôi đã và
đang và vẫn sẽ phải vật lộn với rất nhiều bài tập.

Có lẽ tôi sẽ dừng lại việc đưa ra thêm định nghĩa ở đây, bởi vì bài viết này không mong muốn được
làm một bài giảng toán học. Nếu các bạn muốn biết thêm thì đọc sách có lẽ sẽ tốt hơn nhiều
so với việc đọc cái này.

Một điều tôi muốn nói ở đây là, một trong những câu hỏi thường xuyên nhất
mà mọi người (và chính bản thân tôi ngày xưa) thường đặt ra là: ``Tại sao người ta
lại định nghĩa những thứ này như vậy?'' hoặc ``Tại sao người ta lại quan tâm tới
những thứ này?''

Câu trả lời của tôi bây giờ cho các bạn là: Tât cả những định nghĩa này được đặt ra đều có mục
đích của chúng cả. Có những định nghĩa phải mất tới hàng trăm năm (200, theo như tôi đã
từng đọc được trong một cuốn sách nào đó) thì người ta mới có thể đưa ra được
một định nghĩa ``đúng''.

Tôi không viết nhầm đâu, mất rất nhiều thời gian thì mới có thể đưa ra được \textbf{định nghĩa đúng}.

Càng học sâu thì tôi mới càng thấy rằng, việc xây dựng được hệ thống còn quan trọng hơn
cả việc được ra được hay chứng minh được định lý này nọ. Khi ta có được một hệ thống
lý thuyết và các công cụ tốt, thì các định lý và kết quả hữu dụng sẽ đến một cách rất tự
nhiên. Không cần phải có thiên tài nào để ``tìm ra'', hoặc ``phát mình ra'' những định lý
nào cả.

\section*{Quay trở lại một chút về đời thường và công việc}

bởi vì suy cho cùng thì tôi đang chia sẻ cho anh chị em ở trong công ty.

Tôi thấy cuộc sống và công việc cũng mang tinh thần này. Nếu như chúng ta có một hệ thống
để làm việc và học hỏi tốt, thì những thành công trong công việc sẽ tới một cách rất tự nhiên.
Tôi không cần phải là thiên tài thì mới có thể làm việc tốt được, tôi chỉ cần xây dựng cho
mình một hệ thống tốt.

Cảm ơn các bạn đã đọc đến đây. Chia sẻ một chút vậy thôi, vì đây không phải là blog cá nhân.

\end{document}